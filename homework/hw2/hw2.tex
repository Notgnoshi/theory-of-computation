\documentclass{article}
\input{../homework.sty}

\title{Homework 2}
\author{Austin Gill}
\date{February 15, 2019}

\begin{document}
\maketitle

\section{}
\begin{quote}
    Let $\Sigma = \{a, b\}$ and $L = \{aa, bb\}$. Describe $\overline L$ with set notation.
\end{quote}

Recall that $\overline L = \Sigma^* \setminus L$. Here, $\Sigma^*$ is the set of all strings formed
from $a$ and $b$, including the empty string $\lambda$. Thus $\overline L = \big \{w \in \Sigma^*
    \mid w \notin \{aa, bb\} \big \}$.

\section{}
\begin{quote}
    Find a grammar for the language $L = \{ a^n \mid n \in 2\N \}$
\end{quote}

Clearly, $G = \big(\{S\}, \{a\}, S, P\big)$. The production rules $P$, are then
\begin{align*}
    S & \to aS      \\
    S & \to \lambda
\end{align*}

\section{}
\begin{quote}
    What language does the grammar with these productions generate?
    \begin{align*}
        S & \to Aa \\
        A & \to B  \\
        B & \to Aa
    \end{align*}
\end{quote}

First, we have $G = \big(\{S, A, B\}, \{a\}, S, P\big)$. Second, note that none of the above
production rules result in a terminal symbol. This means there are no strings that are accepted by
the grammar. Thus the language $L(G) = \emptyset$.

Either that or the language consists of one single element, an infinite string of $a$'s. But this
disagrees with my current understanding that languages only contain finite strings.

\section{}
\begin{quote}
    Find a regular expression over the alphabet $\{0, 1\}$ to describe all bitstrings without
    leading zeros (except for 0 itself). So the language is the set
    \[ \{0, 1, 10, 11, 100, \dots \} \]
\end{quote}

\[ 0 + 1 \cdot {(0 + 1)}^*\]

\section{}\label{prob:5}
\begin{quote}
    Find the quintuple for the DFA in \autoref{fig:5:dfa}.
\end{quote}

\begin{figure}[h]
    \centering
    \begin{tikzpicture}[
            ->,
            > = stealth,
            shorten > = 1pt,
            auto,
            node distance = 2.8cm,
            on grid,
            semithick
        ]

        \node[state, initial] (q0) {$q_0$};
        \node[state, right of=q0] (q1) {$q_1$};
        \node[state, accepting, above right of=q1] (q3) {$q_3$};
        \node[state, accepting, below right of=q1] (q2) {$q_2$};

        % \begin{noindent}
        \path (q0) edge[loop above]    node        {$b$} (q0)
                   edge                node        {$a$} (q1)
              (q1) edge                node        {$a$} (q3)
                   edge[bend left=15]  node        {$b$} (q2)
              (q2) edge[bend left=15]  node        {$a$} (q1)
                   edge[bend right=10] node[right] {$b$} (q3)
              (q3) edge[bend right=10] node[left]  {$a$} (q2)
                   edge[loop above]    node        {$b$} (q3);
        % \end{noindent}
    \end{tikzpicture}
    \caption{The DFA for \autoref{prob:5}}\label{fig:5:dfa}
\end{figure}

\section{}
\begin{quote}
    Give a regular expression for the language
    \[L = \{a^n b^m \mid n < 4, m \leq 4\} \]
\end{quote}

\section{}
\begin{quote}
    Describe the language for the regular expressions
    \begin{enumerate}
        \item $a + b$
        \item $a + b^*$
        \item $ab^* + bc^*$
    \end{enumerate}
    Construct an NFA for the regular expression
    \[r = {(a + b)}^*\]
\end{quote}

\end{document}
