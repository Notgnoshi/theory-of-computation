\documentclass{article}
\input{../homework.sty}

\title{Homework 4}
\author{Austin Gill}
\date{April 23, 2019}

\begin{document}
\maketitle

The following problems forbid using the languages given in Example 11.3. Before we can proceed, we must make several definitions.

\begin{defn}
    A grammar $G = (V, T, S, P)$ is said to be \textbf{linear} if every production has at most one variable on the right hand side.
\end{defn}

\begin{defn}
    A grammar $G = (V, T, S, P)$ is \textbf{context-free} if all productions in $P$ have the form \[A \to x\] where $A \in V$ and $x \in {(V \cup T)}^*$.
\end{defn}

\begin{defn}
    A context-free language $L$ is said to be \textbf{linear} if there exists a linear context-free grammar $G$ such that $L = L(G)$.
\end{defn}

\begin{defn}
    A language $L$ is said to be \textbf{deterministic context-free} if and only if there exists a deterministic pushdown accepter $M$ such that $L = L(M)$.
\end{defn}

\section{}
\begin{quote}
    Find two examples of languages that are linear but not deterministic context free.
\end{quote}

The book states that the language $L = \{ww^R\}$ is nondeterministic, but a linear grammar for it is given in Example 5.1.

Using the argument made in Example 7.11, we can claim that $L = \{a^n b^n \} \cup \{a^{3n} b^n\}$ is clearly linear but not deterministic.

\section{}
\begin{quote}
    Find two examples of languages that are deterministic context free but not linear.
\end{quote}

A proof is given in the book that the language $L = \{ a^n b^n a^m b^m \mid n \geq 0, m \geq 0 \}$ is nonlinear yet deterministic context free. Notice that $L$ is the language $\{a^n b^n \mid n \geq 0 \}$ concatenated with itself.

A proof is also given in the book that the language of properly matched brackets given by the grammar $S \to aSa \mid SS \mid \lambda$ is deterministic context-free and nonlinear.
\end{document}
