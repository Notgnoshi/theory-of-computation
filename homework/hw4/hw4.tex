\documentclass{article}
\usepackage{algorithm}
\usepackage{algpseudocode}
\usepackage{amsmath}
\usepackage{amssymb}
\usepackage{amsthm}
\usepackage[titletoc]{appendix}
\usepackage{array}
\usepackage[english]{babel}
\usepackage{booktabs}
\usepackage{cancel}
\usepackage{color}
\usepackage{eqparbox}
\usepackage{float}
\usepackage[margin=1in]{geometry}
\usepackage{graphicx}
\usepackage[hidelinks]{hyperref}
% *must* be loaded after hyperref
\usepackage[toc, acronym, numberedsection=nameref]{glossaries}
\usepackage[utf8]{inputenc}
\usepackage{lipsum}
\usepackage{mathtools}
\usepackage[cache=false]{minted}
\usepackage{parskip}
\usepackage{pgfplots}
\usepackage{scalerel}
\usepackage{skull}
\usepackage{subcaption}
\usepackage{titling}
\usepackage{textcomp}
\usepackage{tikz}
\usepackage[compact, explicit]{titlesec}
\usepackage{textcomp}
\usepackage[nottoc]{tocbibind}
\usepackage[textsize=small]{todonotes}
\usepackage[normalem]{ulem}

% Document Settings

\definecolor{__minted_background_color}{rgb}{0.95, 0.95, 0.98}
\definecolor{__minted_highlight_color}{rgb}{0.88, 0.88, 1.0}
\setminted{autogobble=true,
           style=tango,
           breaklines,
           bgcolor=__minted_background_color,
           highlightcolor=__minted_highlight_color,
           mathescape, % Escape math mode everywhere.
           texcomments,  % Enable latex code inside of comments. Useful for referencing equations.
    }

\usetikzlibrary{arrows, automata, shapes, positioning}
\pgfplotsset{compat=1.16}
\numberwithin{equation}{section}
% Sets the width of the margin TODO notes
\setlength{\marginparwidth}{0.84in}
\reversemarginpar{}

% hex #184c9a
\definecolor{__glossary_entry_color}{rgb}{0.094, 0.298, 0.604}
\renewcommand{\glstextformat}[1]{\textbf{\textcolor{__glossary_entry_color}{#1}}}

% Add glos: to the beginning of the glossary labels.
\renewcommand*{\glsautoprefix}{glos:}

% All I want is to have comment italicized, but I cant figure out how
% to properly modify the existing \Comment macro.
% \algrenewcomment[1]{\hfill\eqparbox{COMMENT}{\textit{// #1}}}
\algnewcommand{\IComment}[1]{\Comment{\textit{#1}}}

% TODO: Should this path be relative to the document root or this file?
\graphicspath{{./figures/}}

% Document Definitions

\newcommand{\C}{\mathbb{C}}
\newcommand{\R}{\mathbb{R}}
\newcommand{\Z}{\mathbb{Z}}
\newcommand{\N}{\mathbb{N}}
\renewcommand{\O}{\mathcal{O}}

\theoremstyle{definition}
\newtheorem{defn}{Definition}[section]

\theoremstyle{plain}
\newtheorem{thm}{Theorem}[section]

\renewcommand{\qedsymbol}{$\skull$}

% An inline TODO command. Doesn't play nicely with \todotableofcontents
\newcommand\todoinline[2][]{\todo[inline, caption={TODO}, #1]{
\begin{minipage}{\textwidth-4pt}#2\end{minipage}}}

% Draw clouds around things. Useful in mathematical proofs.
\newcommand{\cloud}[4][\dots]{%
    \raisebox{-0.4\height}{%
        \begin{tikzpicture}
            \node [cloud,
                   draw,
                   cloud puffs=#2,
                   cloud ignores aspect,
                   minimum height=#3,
                   minimum width=#4] {#1};
        \end{tikzpicture}
    }
}

% Use \ceil*{} or \floor*{}
\DeclarePairedDelimiter{\ceil}{\lceil}{\rceil}
\DeclarePairedDelimiter{\floor}{\lfloor}{\rfloor}

\AtBeginDocument{%
\renewcommand{\sectionautorefname}{Problem}
}

% make each \section a problem.
\titleformat{\section}[runin]{\large\bfseries}{}{0pt}{\titlerule[1.5pt]\newline\vspace*{-4pt}
Problem\quad\thesection\newline}[\vspace{0.01ex}{\titlerule[1.5pt]}]


\title{Homework 4}
\author{Austin Gill}
\date{April 23, 2019}

\begin{document}
\maketitle

The following problems forbid using the languages given in Example 11.3. Before we can proceed, we must make several definitions.

\begin{defn}
    A grammar $G = (V, T, S, P)$ is said to be \textbf{linear} if every production has at most one variable on the right hand side.
\end{defn}

\begin{defn}
    A grammar $G = (V, T, S, P)$ is \textbf{context-free} if all productions in $P$ have the form \[A \to x\] where $A \in V$ and $x \in {(V \cup T)}^*$.
\end{defn}

\begin{defn}
    A context-free language $L$ is said to be \textbf{linear} if there exists a linear context-free grammar $G$ such that $L = L(G)$.
\end{defn}

\begin{defn}
    A language $L$ is said to be \textbf{deterministic context-free} if and only if there exists a deterministic pushdown accepter $M$ such that $L = L(M)$.
\end{defn}

\section{}
\begin{quote}
    Find two examples of languages that are linear but not deterministic context free.
\end{quote}

The book states that the language $L = \{ww^R\}$ is nondeterministic, but a linear grammar for it is given in Example 5.1.

Using the argument made in Example 7.11, we can claim that $L = \{a^n b^n \} \cup \{a^{3n} b^n\}$ is clearly linear but not deterministic.

\section{}
\begin{quote}
    Find two examples of languages that are deterministic context free but not linear.
\end{quote}

A proof is given in the book that the language $L = \{ a^n b^n a^m b^m \mid n \geq 0, m \geq 0 \}$ is nonlinear yet deterministic context free. Notice that $L$ is the language $\{a^n b^n \mid n \geq 0 \}$ concatenated with itself.

A proof is also given in the book that the language of properly matched brackets given by the grammar $S \to aSa \mid SS \mid \lambda$ is deterministic context-free and nonlinear.
\end{document}
